\documentclass[a4paper]{article}
\usepackage[utf8]{inputenc}
\usepackage[russian]{babel}
\usepackage{listings}
\usepackage[a4paper]{geometry}
\usepackage{indentfirst}
\usepackage{graphicx}
\usepackage{caption}

\begin{document}

\title{Лабораторная работа 1 по курсу <<Нелинейная динамика и её приложения>>. \\Отчёт.}
\author{Владислав Соврасов\\ 381503м4}
\date{}
\maketitle

\section{Сравнение скорости сходимости методов поиска корня полинома}
Для поиска корня полинома \(f(x)=x^3+x-1\) были использованы методы дихотомии и
Ньютона. Первый метод обладает линейной сходимостью, а второй --- квадратичной
(т.к. в данном случае производная \(f(x)\) нигде не обращается в \(0\)).
В качестве начального приближения для метода Ньютона была
выбрана точка \(x_0=1\). В методе дихотомии начальный отрезок был взят \([0,5;1]\).
Точность обоих методов: \(10^{-3}\).

\begin{figure}[ht]
	\center
  \includegraphics[width=0.75\textwidth]{../pictures/lab1_convergence.png}
  \caption{Зависимость абсолютного значения полинома от номера итерации метода}
  \label{fig:convergence}
\end{figure}

В результате запуска методов было получено приближённое значение корня
\(\widetilde{x}=0.6816\) (значение округлено). Из рис. \ref{fig:convergence} видно, что метод Нтьютона
сходится за \(4\) итерации, в то время как методу дихотомии необходимо \(9\) итераций для
достижения той же точности.

\section{Построение зависимости корня полинома от параметра}
Необходимо построить зависимость \(x^*(\alpha)\) координаты единственного корня полинома
\(f(x)=x^{N+1}+x+\alpha\) от \(\alpha\) при \(N=2,4,6\) на отрезке \([0;10]\).
Из качественного анализа положения корня следует, что при малых \(\alpha\) зависимость
схожа с линейной, а при больших --- близка к функции \(\sqrt[N+1]{\alpha}\).

Глядя на рис. \ref{fig:roots}, можно убедиться в справедливости качественных оценок.

\begin{figure}[ht]
	\center
  \includegraphics[width=0.75\textwidth]{../pictures/lab1_roots.png}
  \caption{Зависимость корня полинома \(f(x)\) от параметра при различных значениях степени}
  \label{fig:roots}
\end{figure}

\section{Поиск бифуркационного значения параметра системы}
При рассмотрении системы
\begin{displaymath}
	\left\{
  \begin{array}{lr}
    \dot x = \frac{\alpha}{1+y^2}\\
		\dot y = \frac{\alpha}{1+x^2}
  \end{array}
\right.
\end{displaymath}
было выяснено, что критерием бифуркации является изменение количества корней у
полинома \(g(x)=x^5-\alpha x^4 +2x^3 - 2\alpha x^2 + (\alpha^2+1)x-\alpha\).
Необходимо обнаружить бифуркацию численно и построить зависимость корней \(g(x)\)
от \(\alpha\).

Зависимость была построена для \(\alpha \in [0;10]\). При \(\alpha=10\) с помощью
запусков метода Ньютона из различных псевдослучайно сгенерированных точек были найдены
все корни \(g(x)\). Затем, при движении по сетке в направлении \(\alpha=0\),
значения корней в предыдущем узле сетки использовались как начальные приближения
в методе Ньютона для поиска корней в текущем узле.

Как видно из рис. \ref{fig:roots2}, в окрестности точки \(\alpha=2\) происходит
бифуркация и одно состояние равновесия системы превращается в три, которые удаляются
друг от друга с ростом \(\alpha\).

\begin{figure}[ht]
	\center
  \includegraphics[width=0.75\textwidth]{../pictures/lab1_bifurcation.png}
  \caption{Зависимость корней полинома \(g(x)\) от параметра}
  \label{fig:roots2}
\end{figure}

\section{Исходный код}

\lstinputlisting[language=Python, numbers=left]{../scripts/lab1.py}

\end{document}
