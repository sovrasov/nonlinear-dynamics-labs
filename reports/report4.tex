\documentclass[a4paper]{article}
\usepackage[utf8]{inputenc}
\usepackage[russian]{babel}
\usepackage{listings}
\usepackage[a4paper]{geometry}
\usepackage{indentfirst}
\usepackage{graphicx}
\usepackage{caption}
\usepackage{float}
\usepackage{amssymb}

\begin{document}

\title{Лабораторная работа 4 по курсу <<Нелинейная динамика и её приложения>>. \\Отчёт.}
\author{Владислав Соврасов\\ 381503м4}
\date{}
\maketitle

\section{Построение гистограммы распределения собственных чисел случайных матриц}
Рассмотрим матрицы \(A\in \mathbf{R}^{N\times N}\) такие, что \(A=A^T\) и
\begin{displaymath}
	a_{i,j} = \left\{
  \begin{array}{lr}
    2\mathcal{N}(0,1), i=j\\
		\mathcal{N}(0,1), i	\neq j
  \end{array}
\right.
\end{displaymath}
Известно, что такие матрицы (Gaussian Orthogonal Ensemble) имеют \(N\) действительных собственных чисел, которые
распределены по закону \(\rho(\lambda / \sqrt{N})=c \sqrt{4-\lambda^2/N}\), где
\(c\) --- нормировочная константа.

Для построения гистограммы распределения нормированных на \(\sqrt{N}\) собственных чисел были
сгенерированы 100 реализаций случайной матрицы при \(N=1000\). На рис. \ref{fig:goe_eig}
показана нормированная гистограмма и теоретическое распределение (пунктирный график), константа
в котором найдена из условия совпадения максимального значения функции распределения
с максимальным значением гистограммы.

Пусть теперь матрицы \(B\in \mathbf{C}^{N\times N}\) такие, что \(B=B^\dag\) и
\(b_{i,j} = \mathcal{N}(0,1) + i\mathcal{N}(0,1)\) (Gaussian Unitary Ensemble).
Их собственные числа действительные и распределены по тому же закону, что и собственные числа матриц \(A\).
На рис. \ref{fig:gue_eig} можно увидеть гистограмму нормированных собственных чисел,
полученную, как и для матриц \(A\), по сотне реализаций при \(N=1000\). Из рис.
\ref{fig:gue_eig} также можно видеть, что гистограмма достаточно точно совпадает с
теоретическим распределением.
\begin{figure}
	\center
	\includegraphics[width=0.75\textwidth]{../pictures/lab4_goe_eig_hist.pdf}
	\caption{Распределение нормированных собственных чисел для GOE}
	\label{fig:goe_eig}
\end{figure}
\begin{figure}
	\center
	\includegraphics[width=0.75\textwidth]{../pictures/lab4_gue_eig_hist.pdf}
	\caption{Распределение нормированных собственных чисел для GUE}
	\label{fig:gue_eig}
\end{figure}

\section{Построение гистограммы расщепления уровней энергии случайных матриц}
Пусть имеется упорядоченный набор собственных чисел матрицы GOE или GUE:
\begin{displaymath}
\lambda_1 \leqslant \lambda_2 \leqslant \dots \leqslant \lambda_{N-1} \leqslant \lambda_N
\end{displaymath}
Расщепение уровней энергии определяется следующим образом:
\(s_i = \lambda_{i+1}-\lambda_i, i=\overline{1,N-1}\). Для того, чтобы распределение
расщеплений не зависело от размера матрицы, как и для собственных чисел, вводится
нормировка: \(\bar{s_i}=s_i/<s>\), где \(<s>\) --- средняя величина расщепления.

Теоретическое распределений расщеплений для матриц GOE имеет вид:
\(p_{GOE}(\bar{s}) = c_1 \bar{s} \exp(-\frac{\pi}{4}\bar{s}^2)\). На рис. \ref{fig:goe_spacing}
изображена нормированная гистограмма, построенная по полученным в предудущем пункте
реализациям. Для сравнения на том же графике также построена функция \(p_{GOE}(\bar{s})\)
с таким значением \(c_1\), что её максимальное значение совпадает с максимумом гистограммы.
Из графика видно, что гистограмма затухает с увеличением \(\bar{s}\) не так быстро,
как \(p_{GOE}(\bar{s})\), но тем не менее общий качественный вид зависимости совпадает.
В окрестности нуля значения гистограммы растут линейно.

Для матриц GUE распределений расщеплений имеет вид:
\(p_{GUE}(\bar{s}) = c_2 \bar{s}^2 \exp(-\frac{4}{\pi}\bar{s}^2)\). На рис. \ref{fig:gue_spacing}
показана нормированная гистограмма расщеплений и график функии \(p_{GUE}(\bar{s})\) c
вычисленной ранее описанным способом константой \(c_2\). Как и для матриц GOE,
гистограмма затухает медленнее, чем теоретическое распределение, но качественно
похожа на график реального распределения. В частности, виден квадратичный порядок
роста значений гистограммы в окрестности нуля.

\begin{figure}[H]
	\center
	\includegraphics[width=0.75\textwidth]{../pictures/lab4_goe_spacing_hist.pdf}
	\caption{Распределение нормированных расщеплений уровней для GOE}
	\label{fig:goe_spacing}
\end{figure}

\begin{figure}[H]
	\center
	\includegraphics[width=0.75\textwidth]{../pictures/lab4_gue_spacing_hist.pdf}
	\caption{Распределение нормированных расщеплений уровней для GUE}
	\label{fig:gue_spacing}
\end{figure}


\section{Исходный код}
\lstinputlisting[language=Python, numbers=left]{../scripts/lab4.py}

\end{document}
